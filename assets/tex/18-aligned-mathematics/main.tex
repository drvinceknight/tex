\documentclass{article}

\usepackage{amsmath}

\title{Aligned mathematics}
\author{V Knight}
\date{\today}

\begin{document}

\maketitle

Some examples of aligned mathematics:

\begin{align}
    (x+h)^2-x^2 & =x^2+2xh+h^2-x^2 \nonumber\\
                & =2xh+h^2 \nonumber\\
                & =h(2x+h) \nonumber
\end{align}


Some explanatory text:

\begin{align}
    (x+h)^2-x^2 & = x^2+2xh+h^2-x^2 && \text{(by distributivity)}\\
                & = 2xh+h^2         && \text{(by subtraction)}\\
                & = h(2x+h)         && \text{(by factorisation)}
\end{align}
\end{document}
